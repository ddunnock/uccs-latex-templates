% !TeX program = xelatex
% ~/projects/uccs-me-syse-latex/templates/homework_template.text

%%%%%%%%%%%%%%%%%%%%%%%%%%%%%%
%  <COURSE ID>  Homework Template  ––  UCCS College of Engineering & Applied Science
%  Generated: \today
%%%%%%%%%%%%%%%%%%%%%%%%%%%%%%

% NOTE ▸ One‑file template aligned stylistically with the report template.
% ▸ Uses identical Fira Sans font set and DoD‑blue heading color.
% ▸ Copy this file, rename to <COURSEID>_<HW##>.tex, and fill in answers.

\documentclass[12pt]{article}

%------------------------------------------------------------
%  PREAMBLE  (geometry, colors, fonts, shared asset paths)
%------------------------------------------------------------
\usepackage[margin=1in,headheight=14.5pt]{geometry}
\usepackage{array,tabularx,longtable,graphicx}
\usepackage{amsmath,amssymb,mathtools}
\usepackage{booktabs}
\usepackage{fancyhdr}
\usepackage{titlesec}
\usepackage{setspace}
\usepackage{parskip}
\usepackage{enumitem}
\usepackage[table]{xcolor}
\usepackage{lipsum}

% ---------- Shared asset search path ----------
\makeatletter
\def\input@path{{../../../texmf/}}
\graphicspath{{../../../texmf/images/}}
\makeatother
% ----------------------------------------------

% ---------- Corporate palette -----------------
\definecolor{dodblue}{RGB}{0,102,204}  % heading and rule color
\definecolor{headerblue}{RGB}{222,230,242}

% ---------- Fonts -----------------------------
\usepackage{fontspec}
% Body text – Fira Sans Regular/Bold/Italic
\setmainfont[
  Path       = ../../../texmf/fonts/otf/,
  Extension  = .otf,
  UprightFont= *-Regular,
  BoldFont   = *-Bold,
  ItalicFont = *-Italic
]{FiraSans}
% Heading font – light weight variant
\newfontfamily\HeadingFont[
  Path       = ../../../texmf/fonts/otf/,
  Extension  = .otf
]{FiraSans-Light}
% ----------------------------------------------

% ---------- Heading format --------------------
\setcounter{secnumdepth}{1}
\titleformat{\section}
  {\color{dodblue}\HeadingFont\Large\bfseries}
  {\thesection\quad}{0pt}{}
\titlespacing*{\section}{0pt}{1.5\baselineskip}{0.5\baselineskip}
% ----------------------------------------------

% ---------- Page header/footer ----------------
\pagestyle{fancy}
\fancyhf{}
% — left header – student name placeholder
\lhead{<Student Name>}
% — right header – course & term placeholder
\rhead{<Term / Year> – <COURSE ID>}
% — footer page number
\cfoot{\thepage}
% — ensure headers/footers appear on all pages including title page
\fancypagestyle{plain}{%
  \fancyhf{}
  \lhead{<Student Name>}
  \rhead{<Term / Year> – <COURSE ID>}
  \cfoot{\thepage}
}
% ----------------------------------------------

% ---------- Problem environment ---------------
\newenvironment{problem}{\color{dodblue}\itshape}{\par}
% ----------------------------------------------

% ---------- Misc styling ----------------------
\doublespacing
\setlist[itemize]{noitemsep, topsep=0pt}
% ----------------------------------------------

%------------------------------------------------------------
%  USER‑CONFIGURABLE MACROS
%------------------------------------------------------------
\newcommand{\homeworknum}{01}  % ⇦ update per assignment

%------------------------------------------------------------
%  BEGIN DOCUMENT
%------------------------------------------------------------
\begin{document}

%-------------------------------------------------
%  TITLE BLOCK (compact – no separate page)
%-------------------------------------------------
\begin{center}
    % UCCS signature logo
    \includegraphics[width=0.9\textwidth,keepaspectratio]{uccs-logo.png}\\[8\baselineskip]

    {\HeadingFont\fontsize{24}{26}\selectfont\textbf{<COURSE ID>}}\\[0.25\baselineskip]
    {\large <Course Title>}\\[0.15\baselineskip]
    {\small <Professor / Instructor>}\\[2\baselineskip]
\vfill

    {\HeadingFont\fontsize{20}{22}\selectfont\textbf{<Module Number>: <Assignment Title>}}\\[0.5\baselineskip]
    {Submitted by \textbf{<Student Name>}}\\[0.15\baselineskip]
    {\today}
\end{center}

% Optional blank line before problems begin
\newpage

%=================================================
%  PROBLEM 1
%=================================================
\section*{Problem 1 –– Sample Analytical Question}
\begin{problem}
\lipsum[1]
\end{problem}

\textbf{Answer:}\\
\lipsum[2]

A representative formula used in the solution could be written as
\begin{equation}\label{eq:newton}
F = m\,a
\end{equation}
where $F$ is force, $m$ is mass, and $a$ is acceleration.

To illustrate summation:
\begin{equation}
S = \sum_{k=1}^{n} k = \frac{n(n+1)}{2}.
\end{equation}

%=================================================
%  PROBLEM 2
%=================================================
\section*{Problem 2 –– Computational Exercise}
\begin{problem}
\lipsum[3]
\end{problem}

\textbf{Answer:}\\
\lipsum[4]

The governing quadratic formula is
\begin{equation}
\label{eq:quadratic}
x = \frac{-b \pm \sqrt{b^{2}-4ac}}{2a}.
\end{equation}

For continuous compounding, we employ
\begin{equation}
\label{eq:exp}
A(t) = A_0 e^{rt},
\end{equation}
where $A_0$ is the principal and $r$ the nominal rate.

%=================================================
%  PROBLEM 3
%=================================================
\section*{Problem 3 –– Short Design Discussion}
\begin{problem}
\lipsum[5]
\end{problem}

\textbf{Answer:}\\
\lipsum[6]

An example matrix formulation:
\begin{equation}
\mathbf{K}\,\mathbf{x} = \mathbf{f},
\end{equation}
where $\mathbf{K}$ is the stiffness matrix, $\mathbf{x}$ the displacement vector, and $\mathbf{f}$ the load vector.

Eigen‑analysis for vibration modes follows
\begin{equation}
\det\bigl(\mathbf{K} - \omega^{2}\mathbf{M}\bigr) = 0.
\end{equation}

%=================================================
%  PROBLEM 4
%=================================================
\section*{Problem 4 –– Data Interpretation}
\begin{problem}
\lipsum[7]
\end{problem}

\textbf{Answer:}\\
\lipsum[8]

A sample statistical expression:
\begin{equation}
\mu = \frac{1}{N}\sum_{i=1}^{N} x_i, \qquad \sigma^{2} = \frac{1}{N-1}\sum_{i=1}^{N}(x_i-\mu)^{2}.
\end{equation}

Linear regression is obtained by solving
\begin{equation}
\hat{\beta} = (\mathbf{X}^{\top}\mathbf{X})^{-1}\mathbf{X}^{\top}\mathbf{y}.
\end{equation}

%=================================================
%  End of assignment
%=================================================

\end{document}
%%%%%%%%%%%%%%%%%%%%%%%%%%%%%%
%  END –– UCCS Homework Template with Sample Problems
%%%%%%%%%%%%%%%%%%%%%%%%%%%%%%
